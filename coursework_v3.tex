\documentclass[a4paper,12pt]{article}
\usepackage[utf8]{inputenc}
\usepackage[russian]{babel}
\usepackage{amsmath, amssymb}
\usepackage{graphicx}
\usepackage{float}
\usepackage{geometry}
\geometry{top=2cm,bottom=2.5cm,left=3cm,right=2.5cm}

\usepackage[colorlinks=true, urlcolor=blue, linkcolor=black]{hyperref}

\newpage\begin{document}

\thispagestyle{empty} % без номера страницы

\vspace*{3cm}

\begin{center}
\large \textbf{Курсовой проект} \\[1em]
\large По дисциплине: \\«Радиоприёмные устройства» \\[1em]
\Largez \textbf{Разработка приёмника базовой станции UMTS} \\[3em]
\end{center}

\vfill % всё, что после этого, прижмётся к низу страницы

\begin{flushright}
Выполнил:    \qquad\qquad\qquad \qquad\qquad \\
Проверил:      \qquad\qquad\qquad\qquad\qquad \\[3em]
\end{flushright}

\begin{center}
Москва 2025 г.
\end{center}

\newpage
\tableofcontents
\newpage

\section{Техническое задание}
Разработать радиоприёмное устройство базовой станции стандарта UMTS (универсальной мобильной телекоммуникационной системы с прямым расширением спектра), используя техническое задание, представленное в таблице \ref{tab:t1}.

\begin{table}[H]
\centering
\caption{Технические характеристики}
\label{tab:t1}
\begin{tabular}{|l|l|}
\hline
\textbf{Параметр} & \textbf{Значение} \\ \hline
Принцип дуплексирования & Частотное разделение, разнос 95 МГц \\ \hline
Диапазон приёма & 1710--1785 МГц \\ \hline
Диапазон передачи & 1805--1880 МГц \\ \hline
Модуляция & QPSK \\ \hline
Шаг перестройки по частоте & 5 МГц, дискрет 200 кГц \\ \hline
Точность частоты & $\pm$0.05 ppm (Wide Area BS) \\ \hline
BER & $\leq$ 0.001 \\ \hline
Eb/N0 & $\geq$ 5.2 дБ \\ \hline
Чувствительность BS & --121 дБм (12.2 kbps) \\ \hline
Избирательность по соседнему каналу & --115 дБм / --52 дБм, $\pm$5 МГц \\ \hline
Избирательность по побочным каналам & Согласно 3GPP TS 25.101 \\ \hline
Динамический диапазон & --110.7 дБм ... --25 дБм (85.7 дБ) \\ \hline
\end{tabular}
\end{table}

\section{Выбор структуры приёмника}
Для выбора схемы приёмника проведем сравнение возможных вариантов реализации.
\subsection{Супергетеродинный приёмник}
\textbf{Принцип работы:} \newline
ПФ1 совместно с фильтром ПФ2 ослабляет уровень помех по зеркальному и другим побочным каналам. МШУ обеспечивает заданную чувствительность. Побочные продукты преобразования подавляются ФПЧ1.
\newline
\newline
\textbf{Преимущества:}
\begin{itemize}
    \item высокая чувствительность и избирательность,
    \item стабильные параметры.
\end{itemize}
\textbf{Недостатки:}
\begin{itemize}
    \item большое количество компонентов,
    \item сложность микроминиатюризации,
    \item повышенное энергопотребление.
\end{itemize}

\begin{figure}[H]
    \centering
    \includegraphics[width=1\linewidth]{variant_1_sup.png}
    \caption{Схема супергетеродинного приёмника с двукратным преобразованием частоты}
    \label{fig:enter-label}
\end{figure}

\subsection{Приёмник прямого преобразования}
\textbf{Принцип работы:} \newline
квадратурный преобразователь частоты переносит спектр сигнала на две низкочастотные составляющие. УНЧ и ФНЧ выполняют частотную селекцию.   
\newline
\newline
\textbf{Преимущества:}
\begin{itemize}
    \item простота схемы,
    \item минимум внешних компонентов,
    \item возможность реализации в ИМС,
    \item низкое энергопотребление.
\end{itemize}
\textbf{Недостатки:}
\begin{itemize}
    \item утечка гетеродина → постоянная составляющая,
    \item требования к симметрии I/Q каналов,
    \item интермодуляционные искажения.
\end{itemize}
Появление постоянной составляющей на выходе ФНЧ связано в первую очередь с утечкой сигнала гетеродина. В качестве эффективного решения данной проблемы обычно применяют переход на синтезатор частоты с удвоенной рабочей частотой. Частота, равная частоте входного сигнала, получается уже внутри ИМС путем деления на 2, что приводит к почти полному исчезновению излучения через паразитные цепи. Также, правильная компоновка компонентов РЧ блоков, экранирование узлов и применение специальных алгоритмов оценивания в цифровом блоке обработки, помогают устранить большинство недостатков этой схемы.

\begin{figure}[H]
    \centering
    \includegraphics[width=1\linewidth]{variant_2_zero_if.png}
    \caption{Схема приёмника с прямым преобразованием}
    \label{fig:enter-label}
\end{figure}

\subsection{Приёмник с цифровой обработкой на ПЧ}
\textbf{Принцип работы:}\newline
спектр сигнала переносится на ПЧ, затем оцифровывается АЦП и демодулируется цифровыми средствами (DDS, цифровые ФНЧ, I/Q).
\newline
\newline
\textbf{Преимущества:}
\begin{itemize}
    \item идеальная симметрия каналов,
    \item программная гибкость,
    \item многоканальная обработка.
\end{itemize}
\textbf{Недостатки:}
\begin{itemize}
    \item необходимость быстрого АЦП,
    \item повышенное энергопотребление,
    \item сложность и стоимость реализации.
\end{itemize}
\begin{figure}[H]
    \centering
    \includegraphics[width=1\linewidth]{variant_3_digital_if.png}
    \caption{Схема приёмника с цифровой обработкой на ПЧ}
    \label{fig:enter-label}
\end{figure}



\subsection{Выбор схемы}
Исходя из указанных плюсов и минусов структур приёмника была выбрана схема приёмника прямого преобразования. Расширенная схема такого приёмника на рис. 3. За счёт малой элементной базы структура приемника прямого преобразования будет более простой для реализации, а большинство недостатков удастся избежать правильно подобранной элементной базой, различным экранированием, использованием дифференциальных схем гетеродинов и смесителей, а также применением схем (алгоритмов) оценки и компенсации дрейфа постоянной составляющей и не идентичности каналов. За счёт чего удаётся существенно ослабить проблему дрейфа постоянной составляющей сигнала на выходе перемножителей. 

\begin{figure}[H]
    \centering
    \includegraphics[width=1\linewidth]{variant_3_direct_scheme.png}
    \caption{Расширенная схема приёмника прямого преобразования}
    \label{fig:enter-label}
\end{figure}


\section{Расчёт}
\subsection{Заданные параметры}
\begin{itemize}
    \item Мощность полезного сигнала: $P_{DPCH} = -121$ дБм $= -151$ дБВт -  уровень полезного сигнала (канала DPCH) на входе антенны \newline
    Перевод: дБВт= дБм - 30,  т. е. 0 дБм=1 мВт, то 0 дБВт=1 Вт
    \item Коэффициент усиления МШУ: $G_{LNA} = 17$ дБ (50 раз по мощности)
    \item Коэффициент шума (Noise Figure) МШУ: $NF_{LNA} = 1$ дБ
    \item Затухание в полосовом фильтре (Band-Pass Filter): $L_{BPF} = 2.5$ дБ
    \item Запас на неточности реализации цифровой обработки  (Implementation Margin): $L_{IM} = 2$ дБ
    Учитывает все реальные потери, погрешности, допуски, квантизацию, фазовые шумы и т. д.
    \item Коэффициент шума демодулятора: $NF_{DEM} = 11.6$ дБ
    \item Коэффициент усиления демодулятора: $G_{DEM} = 1.5$ дБ
    \item Затухание ANT-RX: $L_{RX} = 2$ дБ  \newline
    Затухание в дуплексере между выводами "ANT - RX" (между антенной и приёмником) в диапазоне частот приёма. То есть сигнал на входе МШУ будет на 2 дБ слабее, чем на антенне.
    \item Затухание TX-RX: $L_{TX/RX} = 35$ дБ  \newline
    Затухание в дуплексере между выводами "TX - RX" (от передатчика к приёмнику)  в диапазоне частот приёма
    \item Полоса: $BW = 3.84$ МГц
    \item Постоянная Больцмана: $k = 1.38\cdot10^{-23}$ Вт/Гц·К
    \item Температура (комнатная): $T = 273$ К
\end{itemize}
Технические характеристики подобранных элементов описаны в Приложении 1

\subsection{Расчёт чувствительности приёмника}
Реальная чувствительность приёмника определяется как минимальный уровень мощности полезного сигнала PDPCH на входе антенны BS, при котором выходная вероятность ошибки на бит BER стандартного измерительного канала передачи данных не превышает заданной величины.
\subsubsection{Исходные данные}
Выпишем и посчитаем данные, которые понадобятся для этого пункта.
\begin{itemize}
\item Мощность полезного сигнала: $P_{DPCH} = -121$ дБм
\item Системный шум от базовой станции:
\begin{equation}
P_{N}^{SYS} = 10 · log_{10}(10^{P_{RX}/10} - 10^{P_{DPCH}/10})= -111.1 \text{дБм}
\end{equation}
\item Шум от передатчика (через дуплексер): $P_{N}^{TX} = -70 - L_{TX/RX} = -105$ дБм
\item Коэффициент шума МШУ: $NF_{LNA} = 1$ дБ
\item Усиление МШУ: $G_{LNA} = 17$ дБ
\item Затухание в полосовом фильтре: $L_{BPF} = 2.5$ дБ
\item Коэффициент шума демодулятора: $NF_{DEM} = 11.6$ дБ
\item Усиление демодулятора: $G_{DEM} = 1.5$ дБ
\item Ширина полосы: $BW = 3.84$ МГц 
\item Постоянная Больцмана: $k = 1.38\cdot10^{-23}$ Вт/Гц·К
\item Температура: $T = 273$ К 
\item Implementation Margin: запас на реализацию цифровой обработки:\\ $L_{IM} = 2$ дБ
\item Processing Gain: энергетический выигрыш вследствие свёртки шумоподобного сигнала PNS (определяется из соотношения ширины спектра сигнала WCDMA и полосы полезного информационного сигнала после свёртки): 
\begin{equation}
G_{PG} = 10  log_{10}\frac{BW}{R_{data}} = 10log_{10}(314.75) = 25 \text{дБ}
\end{equation}
\end{itemize}
\subsubsection{Расчёт теплового шума и суммарного коэффициента шума}
Кроме системных имеется ещё два источника помех:  \\
1) аддитивный белый гауссовский шум (Additive White Gaussian Noise - AWGN), обусловленный собственными тепловыми шумами каскадов усиления приемника (особенно его входного LNA);\\
2) шумовая составляющая шумов передатчика BS в диапазоне принимаемых частот, спектральную плотность которого можно также считать постоянной. Рассчитывается с учетом затухания сигнала в дуплексере и мощности собственных шумов передатчика в диапазоне частот приёма PN(BW) = -70 дБм -- эта величина была получена на основе анализа шумов типовых передатчиков данного диапазона\\

Тепловой шум в полосе $BW$:
\begin{equation}
P_{thermal} = 10 \cdot \log_{10}(kTB) + 30 = 10 \cdot \log_{10}(1.38 \cdot 10^{-23} \cdot 273 \cdot 3.84 \cdot 10^6) + 30 \approx -108.4\ \text{дБм}
\end{equation}

Суммарный коэффициент шума приёмного тракта:
\begin{equation}
NF_\Sigma = 10 \cdot \log_{10} \left(10^{\frac{NF_{LNA}}{10}} + \frac{10^{\frac{NF_{DEM}}{10}} - 1}{10^{\frac{G_{LNA} - L_{BPF}}{10}}} \right) 
\end{equation}

\begin{align}
NF_\Sigma &= 10 \cdot \log_{10} \left(10^{\frac{1}{10}} + \frac{10^{\frac{11.6}{10}} - 1}{10^{\frac{17 - 2.5}{10}}} \right) \notag \\
&= 10 \cdot \log_{10} \left(1.26 + \frac{14.45 - 1}{28.18} \right) \notag \\
&= 10 \cdot \log_{10} (1.26 + 0.477) = 10 \cdot \log_{10} (1.737) \approx 2.4\ \text{дБ}
\end{align}
\\
Суммарный шум приёмника:
\begin{equation}
P_{N0} = P_{thermal} + NF_\Sigma = -108.4\text{дБм} + 2.4 \text{дБм} = -106\ \text{дБм}
\end{equation}

\subsubsection{Расчёт результирующего отношения «сигнал/(шум + помеха)»}
Суммарная помеха:
\begin{equation}
P_{NI} = 10 \cdot \log_{10}\left(10^{\frac{P_{N}^{SYS}}{10}} + 10^{\frac{P_{N0}}{10}} + 10^{\frac{P_{NTX}}{10}}\right)
\end{equation}
Подставляя значения:
\begin{equation}
P_{NI} = 10 \cdot \log_{10}\left(10^{-11.11} + 10^{-10.60} + 10^{-10.5}\right) \approx -104.9\ \text{дБм}
\end{equation}

Отношение сигнал/шум+помеха:
\begin{equation}
\left(\frac{S}{N+I}\right) = P_{DPCH} - P_{NI} = -121 - (-104.9) = -16.1\ \text{дБ}
\end{equation}

\subsubsection{Переход к эффективному значению}
\begin{equation}
\left(\frac{S}{N+I}\right)_{EFF} = \left(\frac{S}{N+I}\right) - L{IM} + G_{PG} = -16.1 - 2 + 25 = 6.9\ \text{дБ}
\end{equation}

\subsubsection{Вывод:} полученное значение $6.9$ дБ превышает требуемое значение $5.2$ дБ с заметным запасом, что подтверждает выполнение условия чувствительности приёмника.

\subsection{Расчёт избирательности по соседнему каналу}
Избирательность по соседнему каналу (ACS – Adjacent Channel Selectivity) является мерой способности приёмника принимать полезный WCDMA сигнал с заданным уровнем качества (величина BER не превышает 10-3) в присутствии мешающего сигнала по соседнему каналу (смещение по частоте на ±5 МГц).
\subsubsection{Исходные данные}
\begin{itemize}
\item Мощность полезного сигнала: $P_{DPCH} = -115$ дБм
\item Уровень мешающего сигнала: $P_{adj} = -52$ дБм
\item Системный шум: $P_{N}^{SYS} = -111.1$ дБм
\item Шум приёмника: $P_{N}^{0} = -101$ дБм
\item Шум от передатчика: $P_{N}^{TX} = -105$ дБм
\item Смещение по частоте: $\pm 5$ МГц
\item Требуемое значение $E_b/N_0$: 5.2 дБ
\item Processing Gain: $G_{PG} = 25$ дБ
\item Implementation Margin: $L_{IM} = 2$ дБ
\end{itemize}

Мешающий сигнал создаёт дополнительную помеху, которая поступает через фильтр нижних частот (ФНЧ). Чтобы приём был возможен, сигнал после ФНЧ должен удовлетворять требованию:
\begin{equation}
\left( \frac{S}{N + I + I_{adj}} \right)_{EFF} \geq 5.2\ \text{дБ}
\end{equation}



\subsubsection{Расчёт отношения «сигнал/(шум+помеха)»}
\begin{equation}
\frac{S}{N + I} = P_{DPCH} - 10 \cdot \log_{10} \left(10^{P_{N}^{SYS}/10} + 10^{P_{N}^{0}/10} + 10^{P_{N}^{TX}/10} \right)
\end{equation}

Подставим значения:
\begin{equation}
\frac{S}{N + I} = -115 - 10 \cdot \log_{10} \left(10^{-11.11} + 10^{-10.1} + 10^{-10.5} \right) \approx -15.8\ \text{дБ}
\end{equation}

Эффективное значение
\begin{equation}
\left( \frac{S}{N + I} \right)_{EFF} = \left( \frac{S}{N + I} \right) + G{PG} - L_{IM} = -15.8 + 25 - 2= 7.2\ \text{дБ}
\end{equation}


\subsubsection{Расчёт отношения «сигнал/помеха» без фильтрации}
\begin{equation}
\left( \frac{S}{I_{ACI}} \right)_{EFF} = P_{DPCH} - P_{ACI} + G_{PG} - L_{IM} = -115 + 52 + 25 - 2 = -40\ \text{дБ}
\end{equation}
Так как нужно добиться показателя $>E_B/N_0 =5.2$ дБ, т. е.  минимального значения эффективного отношения «сигнал/помеха», для обеспечения приёма требуется:
\begin{equation}
\left( \frac{S}{I_{ACI}} \right)_{EFF}^{target} = 6\ \text{дБ}
\end{equation}

Тогда необходимое подавление фильтром:
\begin{equation}
L_{LPF}(\Delta f) = 6 - (-40) = 46\ \text{дБ}
\end{equation}

\textbf{Вывод:} фильтр должен ослаблять мешающий сигнал на частоте $\pm5$ МГц минимум на 46 дБ, чтобы обеспечить требуемое качество приёма.  

\subsubsection{Обоснование выбора цифрового фильтра}
Данную избирательность мы с запасом можем реализовать при помощи микросхемы LTC6603: согласно  даташиту «LTC6603 - Dual Adjustable Lowpass Filter», при настройке фильтра на частоту среза 2,5 МГц:​

На частоте 4 МГц (что составляет 1,6× частоты среза) подавление сигнала достигает –43 дБ.​

Учитывая, что мешающий сигнал находится на смещении ±5 МГц от полезного сигнала, и при соответствующей настройке фильтра, можно ожидать подавление на уровне –46 дБ или выше.


\begin{table}[H]
\centering
\caption{Технические характеристики фильтра LTC6603}
\begin{tabular}{|l|l|}
\hline
\textbf{Параметр} & \textbf{Значение} \\ \hline
Фирма-изготовитель & Linear Technology \\ \hline
Модель & LTC6603 \\ \hline
Тип фильтра & Двухканальный программируемый ФНЧ \\ \hline
Максимальная частота среза & До 2.5 МГц \\ \hline
Порядок фильтра & До 9-го \\ \hline
Диапазон усиления & 0, 6, 12, 24 дБ \\ \hline
Подавление на частоте ±5 МГц & До 60 дБ \\ \hline
Форма сигнала & Линейно-фазовый отклик \\ \hline
Корпус & QFN, 4×4 мм \\ \hline
\end{tabular}
\end{table}


\begin{figure}[H]
    \centering
    \includegraphics[width=1\linewidth]{filter.png}
    \caption{Характеристики чипа LTC6603}
    \label{fig:enter-label}
\end{figure}

Слева на рис.5 - АЧХ. Видно, что 
\begin{itemize}
\item В области ниже 2.5 МГц (настройка LPF1 = 1, BW = 2.5 МГц) сигнал почти не ослабляется (усиление ≈ 0 дБ).
\item На частоте ~5 МГц (это как раз смещение соседнего канала в UMTS) видно:
Усиление падает до –46...–50 дБ, в зависимости от режима.
\item Ниже –60 дБ — ещё сильнее подавляется сигнал при частотах > 6 МГц.
\end{itemize}

\\
Справа на рис.5 - АЧХ + усиление и групповая задержка. Видно, что 
\begin{itemize}
\item Усиление почти не влияет на частотную характеристику.
\item Групповая задержка стабильна до среза, потом быстро растёт.
\item Это важно для ЦСП, так как сильный перекос фаз может ухудшить демодуляцию. Но по графику в пределах до 2.5 МГц — всё гладко.
\end{itemize}
\\
\textbf{Вывод}: На ±5 МГц LTC6603 обеспечивает стабильное подавление $≥ 46$ дБ\\

\subsubsection{Расчёт итогового эффективного отношения «сигнал/(шум + помеха)»}

Рассчитаем новое значение эффективного отношения «сигнал/(шум + помеха)» с учётом подавления ФНЧ в выбранной ИМС , $L_{LPF}(\Delta F) = 60$ дБ:

\begin{equation}
\left( \frac{S}{I_{ACI}} \right)_{EFF} = L_{LPF}(\Delta F) - 40 = 60 - 40 = 20\\ \text{дБ}
\end{equation}

\begin{align}
\left( \frac{S}{N + I_{ACI}} \right)_{EFF} = 10 \cdot \log_{10} \left( \frac{1}{10^{-7.2/10} + 10^{-20/10}} \right) \notag \\
= 10 \cdot \log_{10} \left( \frac{1}{0.1905 + 0.01} \right)
= 10 \cdot \log_{10} (4.93)
\approx 7.3\ \text{дБ}
\end{align}

\subsubsection{Вывод}
Итоговое эффективное отношение $\left( \frac{S}{N + I_{ACI}} \right)_{EFF} = 6.9$ дБ $> E_b/N_0 = 5.2$ дБ. Устройство обеспечивает приём с запасом, который помогает бороться со многими дополнительными факторами (помехами), которые не учитывались в расчёте.



\subsection{Требуемое сквозное усиление приёмника}
\subsubsection{Диапазон входных сигналов}
Согласно стандарту UMTS (3GPP TS 25.101), минимальная мощность входного сигнала, при котором приём должен быть обеспечен — $P_{min} = -115$ дБм. 

Максимальная входная мощность, не вызывающая перегрузку тракта, принимается как $P_{max} = -25$ дБм.

\begin{equation}
D = P_{max} - P_{min} = -25 - (-115) = 90\ \text{дБ}
\end{equation}

Таким образом, требуемый динамический диапазон приёмника составляет 90 дБ.

\subsubsection{Выбор АЦП и допустимые уровни}
Для цифровой обработки используется АЦП с разрядностью 14 бит. Типовое значение эффективного динамического диапазона такого АЦП составляет около 72 дБ.

Примем, что сигнал на входе АЦП должен находиться в диапазоне от:
\begin{itemize}
\item $P_{out_min} = -12$ дБм — минимальный уровень, необходимый для устойчивой работы АЦП
\item $P_{out_max} = -2$ дБм — максимальный уровень, не вызывающий искажений
\end{itemize}

\subsubection{Оценка минимального и максимального усиления тракта}

Для обеспечения требуемого уровня сигнала на выходе, усиление приёмного тракта должно быть в пределах:

\begin{equation}
G_{min} = P_{out_min} - P_{max} = -12 - (-25) = 13\ \text{дБ}
\end{equation}

\begin{equation}
G_{max} = P_{out_max} - P_{min} = -2 - (-115) = 113\ \text{дБ}
\end{equation}

Таким образом, диапазон регулировки усиления должен составлять:

\begin{equation}
D = G_{max} - G_{min} = 113 - 13 = 100\ \text{дБ}
\end{equation}




\subsubsection{Параметры выбранного АЦП}
Во всём динамическом диапазоне на выходе аналоговой части приемного тракта должен быть обеспечен уровень напряжения, примерно соответствующий половине полной шкалы используемого в Baseband процессоре АЦП, этот запас по номинальному уровню в 6дБ (2 раза) необходим для отсутствия искажений сигнала из-за ограничения изменяющейся амплитудной огибающей.  

В качестве АЦП был выбран Analog Devices AD6600 - в таблице приведены его основные характеристики.

\begin{table}[H]
\centering
\caption{Технические характеристики АЦП AD6600}
\begin{tabular}{|l|l|}
\hline
\textbf{Параметр} & \textbf{Значение} \\ \hline
Фирма-изготовитель & Analog Devices \\ \hline
Модель & AD6600 \\ \hline
Дифференциальный вход & + \\ \hline
Уровень дифф. сигнала полной шкалы, V & 2 (пик-пик) \\ \hline
Требуемый запас, дБ & 6 \\  \hline
Тактовая частота & 450 МГц  \\ \hline
Число разрядов & 11 бит \\ \hline
Нагрузка R & 200 Ом \\ \hline
\end{tabular}
\end{table}

\subsubsection{Оценка максимальной мощности на входе АЦП}
АЦП имеет дифференциальный двухканальный вход с уровнем полной шкалы 2 В пик-пик. Для исключения искажений, вызванных амплитудной огибающей сигнала, закладывается запас по уровню 6 дБ. Это соответствует снижению амплитуды в 2 раза, т.е. использование только половины шкалы АЦП.

При этом максимальное выходное напряжение аналогового тракта должно быть:
\begin{equation}
U_{pp} = 1\ \text{В (пик-пик)},\quad U_{peak} = 0.5\ \text{В}
\end{equation}

Эффективное (среднеквадратичное) значение напряжения:
\begin{equation}
    U_{EFF} = \frac{U_{peak}}{\sqrt{2}} = \frac{0.5}{\sqrt{2}} \approx 0.356\ \text{В}
\end{equation}

Мощность сигнала на нагрузке 200 Ом:
\begin{equation}
P = 10 \cdot \log_{10} \left( \frac{U_{eff}^2}{R} \cdot \frac{1}{1~\text{мВт}} \right)
= 10 \cdot \log_{10} \left( \frac{0.356^2}{200 \cdot 0.001} \right) \approx -2\ \text{дБм}
\end{equation}

\textbf{Вывод:} максимальная допустимая мощность сигнала на входе АЦП при учёте запаса составляет около $-2$ дБм. Этот уровень используется в дальнейшем при расчёте максимального усиления тракта.


\subsubsection{Расчёт диапазона усиления приёмного тракта}

\textbf{Оценка требуемого диапазона усиления}

На основании ранее рассчитанных мощностей:

\begin{itemize}
\item Минимальный уровень сигнала: $P_{min} = -115$ дБм
\item Максимальный уровень сигнала: $P_{max} = -25$ дБм
\item Максимальный выходной уровень (до АЦП): $P_{out_{max}} = -2$ дБм
\item Минимальный выходной уровень (до АЦП): $P_{out_{min}} = -1.98$ дБм
\end{itemize}

\begin{equation}
G_{max} = P_{out_{max}} - P_{min} = -2 - (-115) = 113\ \text{дБ}
\end{equation}

\begin{equation}
G_{min} = P_{out_{min}} - P_{max} = -1.98 - (-25) = 23.02\ \text{дБ}
\end{equation}

\begin{equation}
D = G_{min} \ldots G_{max} = 26.98 \ldots 113\ \text{дБ}
\end{equation}

\textbf{Деление тракта на части}

Приёмный тракт делится на:
\begin{itemize}
\item \textbf{Радиотракт} — от антенны до выхода I/Q демодулятора
\item \textbf{Видеотракт} — от выхода демодулятора до входа АЦП
\end{itemize}

\subsubsection{Расчёт усиления радиотракта}

Учитываются следующие компоненты:
\begin{itemize}
\item Усиление МШУ: $G_{LNA} = 15$ дБ
\item Затухание в дуплексере: $L_{RX} = 4$ дБ
\item Затухание в полосовом фильтре: $L_{BPF} = 1.8$ дБ
\item Усиление I/Q демодулятора: $G_{DEM} = 2$ дБ
\end{itemize}

\begin{equation}
K_{рт} = -L_{RX} - L_{BPF} + G_{LNA} + G_{DEM} = -4 - 1.8 + 15 + 2 = 11.2\ \text{дБ}
\end{equation}

\textbf{Добавление усиления видеотракта} 
В видеотракт включён УНЧ LTC6603 с фиксированным усилением до 24 дБ со встроенным фильтром нижних частот (ФНЧ).
\begin{equation}
K_\Sigma = K_{рт} + K_{VGA} = 11.2 + 24 = 35.2\ \text{дБ}
\end{equation}
 
\begin{table}[H]
\centering
\caption{Технические характеристики УНЧ (LTC6603)}
\begin{tabular}{|l|l|}
\hline
\textbf{Параметр} & \textbf{Значение} \\ \hline
Фирма-изготовитель & Linear Technology \\ \hline
Модель & LTC6603 \\ \hline
Диапазон частот до, МГц & 2.5 \\ \hline
Коэффициент шума & \begin{tabular}[c]{@{}l@{}}
Gain = 0 dB: –124 dBm/Hz \\
Gain = 6 dB: –129 dBm/Hz \\
Gain = 12 dB: –135 dBm/Hz \\
Gain = 24 dB: –145 dBm/Hz
\end{tabular} \\ \hline
Усиление, дБ & 0 / 6 / 12 / 24 \\ \hline
\end{tabular}
\end{table}
\textbf{Проблема}

Требуется до 113 дБ усиления, а получено только 35.2 дБ. Недостаёт около 78 дБ.

\textbf{Решение }

Для компенсации недостающего усиления применяется дополнительный УНЧ AD8338 с переменным усилением 0–80 дБ (не менее 76дБ). Он хорошо подходит для реализации АРУ (автоматической регулировки усиления).

\textbf{Вывод:} Один УНЧ (LTC6603) обеспечивает базовое усиление и фильтрацию, а AD8338 — гибкую регулировку. В сумме они обеспечивают диапазон усиления $D = 26.98 \ldots 113$ дБ, что покрывает весь динамический диапазон приёмника.


\vspace{1em}

\begin{table}[H]
\centering
\caption{Технические характеристики дополнительного УНЧ (AD8338)}
\begin{tabular}{|l|l|}
\hline
\textbf{Параметр} & \textbf{Значение} \\ \hline
Фирма-изготовитель & Analog Devices \\ \hline
Модель & AD8338 \\ \hline
Диапазон частот до, МГц & 18 \\ \hline
Шаг изменения усиления, дБ & 0.5 \\ \hline
Усиление, дБ & 0…80 \\ \hline
\end{tabular}
\end{table}

\subsubsection{Обоснование выбора схемы усиления из двух УНЧ}

В структуре приёмного тракта используются два усилительных каскада: основной УНЧ со встроенным фильтром нижних частот (LTC6603) и дополнительный усилитель с переменным усилением (AD8338). Такой подход обусловлен следующими факторами:

\begin{itemize}
\item \textbf{Функциональное разделение каскадов.} Усилитель LTC6603 выполняет не только функцию усиления сигнала, но и обеспечивает необходимую избирательность по соседнему каналу за счёт встроенного линейно-фазового ФНЧ. AD8338 применяется в качестве регулируемого усилителя (VGA), позволяющего адаптировать усиление под текущий уровень входного сигнала.

\item \textbf{Оптимизация шумовых характеристик.} LTC6603 имеет низкий коэффициент шума даже при усилении до 24 дБ, что позволяет предварительно усилить сигнал без существенного увеличения шумов. Это особенно важно на начальных этапах тракта, где уровень сигнала минимален.

\item \textbf{Гибкость и регулировка.} AD8338 позволяет плавно изменять усиление в диапазоне от 0 до 80 дБ с шагом 0.5 дБ. Это делает возможным реализацию автоматической регулировки усиления (АРУ), обеспечивая стабильную амплитуду сигнала на входе АЦП независимо от уровня принимаемого сигнала.

\item \textbf{Минимизация искажений.} Разделение усиления между двумя каскадами позволяет каждому из них работать в своей линейной области, уменьшая вероятность перегрузки и нелинейных искажений.

\item \textbf{Снижение требований к одному компоненту.} Использование двух усилителей позволяет избежать необходимости в одном усилителе с чрезмерно широким диапазоном регулировки, что упростило выбор компонентов и повысило надёжность схемы.

\end{itemize}

Таким образом, применение двухкаскадной схемы усиления обеспечивает требуемый динамический диапазон усиления приёмного тракта, улучшает шумовые характеристики и повышает стабильность работы устройства в условиях реальных помех и изменений уровня сигнала.



\subsubsection{Окончательное сквозное усиление приёмного тракта}

Учитывая второй каскад усиления (УНЧ2 — AD8338), расчёт полного усиления тракта принимает следующий вид:
\begin{equation}
K_\Sigma = K_{рт} + K_{VGA1} + K_{VGA2} = 11.2 + 24 + 80 = 115.2\ \text{дБ}
\end{equation}

Это значение превышает требуемое максимальное усиление $G_{max} = 113$ дБ и обеспечивает необходимый запас.

Для обеспечения минимального коэффициента передачи $G_{min} = 26.98$ дБ, необходимо обеспечить снижение усиления тракта. Примем примерный уровень выходного сигнала в момент минимального усиления $K_{pm} = 39.6$ дБ, тогда:
\begin{equation}
K_{pm} - 30.1 = 9.5\ \text{дБ}
\end{equation}

Минимальное возможное значение усиления должно быть снижено на:
\begin{equation}
|11.4 - 11| = 0.4\ \text{дБ}
\end{equation}

Это снижение может быть реализовано либо за счёт регулировки в демодуляторе, либо через VGA каскад. Демодулятор допускает изменение усиления в пределах от 0 до 15 дБ, что с запасом перекрывает требуемую регулировку в 0.4 дБ.

Таким образом, полная схема приёмника с двумя каскадами видеотракта обеспечивает как максимальное усиление (с запасом), так и необходимую гибкость для регулировки в пределах всего требуемого диапазона усиления.

Управляющее напряжение $u_y$, поступающее на АРУ, формирует соответствующий режим работы активных элементов тракта по постоянному току, обеспечивая динамическую адаптацию усиления в зависимости от входного сигнала.




\newpage

\section{Список используемой литературы}
\begin{enumerate}

\item 	3GPP TS 25.101 Technical Specification Group Radio Access Network
\item  Клич С.М., Кривенко А.С., Носикова Г.Н. и др., Проектирование радиоприёмных устройств: Учебное пособие для вузов / Под ред. А.П. Сиверс. –  М.: Советское радио, 1976 
\item  Логвинов В.В. Радиоприёмные устройства систем мобильной связи: Учебно-методическое пособие – М.: МТУСИ, 2016
\item  Косичкина Т.П. Курсовое проектирование радиоприёмных устройств для телерадиовещания: Учебно-методическое пособие – М.: МТУСИ, 2018
\item  Фомин Н.Н., Буга Н.Н., Головин О.В., и др., Радиоприемные устройства: Учебник для вузов / Под ред. Н.Н.Фомина. – М.: Горячая линия –Телеком, 2007.
\item  Пестряков А.В. Проектирование радиоприёмных устройств мобильной связи: Практические занятия, Москва, 2024.

\item  \href{URLhttps://www.transko.com/Word/RF%20SAW%20Filter/TF-9187-ND.pdf}{Datasheet for TF-9187-ND - Low-loss RF SAW filter for mobile systems}
\item  \href{https://www.analog.com/media/en/technical-documentation/data-sheets/5510fa.pdf}{Datasheet for LTC5510 - 1MHz to 6GHz Wideband High Linearity Active Mixer}
\item  https://www.analog.com/media/en/technical-documentation/data-sheets/hmc382.pdf
\item  https://pdf1.alldatasheet.com/datasheet-pdf/view/265342/LINER/LTC6603.html
\item  https://pdf1.alldatasheet.com/datasheet-pdf/view/48122/AD/AD6600.html
\item  https://pdf1.alldatasheet.com/datasheet-pdf/view/513439/AD/AD8338.html
\end{enumerate}

 \newpage

\section{Приложениe}
\begin{enumerate}
\item Фильтр TF-9187-ND \\
\href{URLhttps://www.transko.com/Word/RF%20SAW%20Filter/TF-9187-ND.pdf}{Datasheet for TF-9187-ND - Low-loss RF SAW filter for mobile systems}
\begin{figure}[H]
    \centering
    \includegraphics[width=1\linewidth]{TF-9187.png}
    \caption{Характеристики фильтра}
    \label{fig:enter-label}
\end{figure}
\end{enumerate}

\end{document}

