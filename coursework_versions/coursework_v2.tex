\documentclass[a4paper,12pt]{article}
\usepackage[utf8]{inputenc}
\usepackage[russian]{babel}
\usepackage{amsmath, amssymb}
\usepackage{graphicx}
\usepackage{float}
\usepackage{geometry}
\geometry{top=2cm,bottom=2.5cm,left=3cm,right=2.5cm}


\newpage\begin{document}

\thispagestyle{empty} % без номера страницы

\vspace*{3cm}

\begin{center}
\large \textbf{Курсовой проект} \\[1em]
\large По дисциплине: \\«Радиоприёмные устройства» \\[1em]
\Largez \textbf{Разработка приёмника базовой станции UMTS} \\[3em]
\end{center}

\vfill % всё, что после этого, прижмётся к низу страницы

\begin{flushright}
Выполнил:    \qquad\qquad\qquad \qquad\qquad \\
Проверил:      \qquad\qquad\qquad\qquad\qquad \\[3em]
\end{flushright}

\begin{center}
Москва 2025 г.
\end{center}

\newpage
\tableofcontents
\newpage

\section{Техническое задание}
Разработать радиоприёмное устройство базовой станции стандарта UMTS (универсальной мобильной телекоммуникационной системы с прямым расширением спектра), используя техническое задание, представленное в таблице \ref{tab:t1}.

\begin{table}[H]
\centering
\caption{Технические характеристики}
\label{tab:t1}
\begin{tabular}{|l|l|}
\hline
\textbf{Параметр} & \textbf{Значение} \\ \hline
Принцип дуплексирования & Частотное разделение, разнос 95 МГц \\ \hline
Диапазон приёма & 1710--1785 МГц \\ \hline
Диапазон передачи & 1805--1880 МГц \\ \hline
Модуляция & QPSK \\ \hline
Шаг перестройки по частоте & 5 МГц, дискрет 200 кГц \\ \hline
Точность частоты & $\pm$0.05 ppm (Wide Area BS) \\ \hline
BER & $\leq$ 0.001 \\ \hline
Eb/N0 & $\geq$ 5.2 дБ \\ \hline
Чувствительность BS & --121 дБм (12.2 kbps) \\ \hline
Избирательность по соседнему каналу & --115 дБм / --52 дБм, $\pm$5 МГц \\ \hline
Избирательность по побочным каналам & Согласно 3GPP TS 25.101 \\ \hline
Динамический диапазон & --110.7 дБм ... --25 дБм (85.7 дБ) \\ \hline
\end{tabular}
\end{table}

\section{Выбор структуры приёмника}
Для выбора схемы приёмника проведем сравнение возможных вариантов реализации.
\subsection{Супергетеродинный приёмник}
\textbf{Принцип работы:} \newline
ПФ1 совместно с фильтром ПФ2 ослабляет уровень помех по зеркальному и другим побочным каналам. МШУ обеспечивает заданную чувствительность. Побочные продукты преобразования подавляются ФПЧ1.
\newline
\newline
\textbf{Преимущества:}
\begin{itemize}
    \item высокая чувствительность и избирательность,
    \item стабильные параметры.
\end{itemize}
\textbf{Недостатки:}
\begin{itemize}
    \item большое количество компонентов,
    \item сложность микроминиатюризации,
    \item повышенное энергопотребление.
\end{itemize}

\begin{figure}[H]
    \centering
    \includegraphics[width=1\linewidth]{variant_1_sup.png}
    \caption{Схема супергетеродинного приёмника с двукратным преобразованием частоты}
    \label{fig:enter-label}
\end{figure}

\subsection{Приёмник прямого преобразования}
\textbf{Принцип работы:} \newline
квадратурный преобразователь частоты переносит спектр сигнала на две низкочастотные составляющие. УНЧ и ФНЧ выполняют частотную селекцию.   
\newline
\newline
\textbf{Преимущества:}
\begin{itemize}
    \item простота схемы,
    \item минимум внешних компонентов,
    \item возможность реализации в ИМС,
    \item низкое энергопотребление.
\end{itemize}
\textbf{Недостатки:}
\begin{itemize}
    \item утечка гетеродина → постоянная составляющая,
    \item требования к симметрии I/Q каналов,
    \item интермодуляционные искажения.
\end{itemize}
Появление постоянной составляющей на выходе ФНЧ связано в первую очередь с утечкой сигнала гетеродина. В качестве эффективного решения данной проблемы обычно применяют переход на синтезатор частоты с удвоенной рабочей частотой. Частота, равная частоте входного сигнала, получается уже внутри ИМС путем деления на 2, что приводит к почти полному исчезновению излучения через паразитные цепи. Также, правильная компоновка компонентов РЧ блоков, экранирование узлов и применение специальных алгоритмов оценивания в цифровом блоке обработки, помогают устранить большинство недостатков этой схемы.

\begin{figure}[H]
    \centering
    \includegraphics[width=1\linewidth]{variant_2_zero_if.png}
    \caption{Схема приёмника с прямым преобразованием}
    \label{fig:enter-label}
\end{figure}

\subsection{Приёмник с цифровой обработкой на ПЧ}
\textbf{Принцип работы:}\newline
спектр сигнала переносится на ПЧ, затем оцифровывается АЦП и демодулируется цифровыми средствами (DDS, цифровые ФНЧ, I/Q).
\newline
\newline
\textbf{Преимущества:}
\begin{itemize}
    \item идеальная симметрия каналов,
    \item программная гибкость,
    \item многоканальная обработка.
\end{itemize}
\textbf{Недостатки:}
\begin{itemize}
    \item необходимость быстрого АЦП,
    \item повышенное энергопотребление,
    \item сложность и стоимость реализации.
\end{itemize}
\begin{figure}[H]
    \centering
    \includegraphics[width=1\linewidth]{variant_3_digital_if.png}
    \caption{Схема приёмника с цифровой обработкой на ПЧ}
    \label{fig:enter-label}
\end{figure}



\subsection{Выбор схемы}
Исходя из указанных плюсов и минусов структур приёмника была выбрана схема приёмника прямого преобразования. Расширенная схема такого приёмника на рис. 3. За счёт малой элементной базы структура приемника прямого преобразования будет более простой для реализации, а большинство недостатков удастся избежать правильно подобранной элементной базой, различным экранированием, использованием дифференциальных схем гетеродинов и смесителей, а также применением схем (алгоритмов) оценки и компенсации дрейфа постоянной составляющей и не идентичности каналов. За счёт чего удаётся существенно ослабить проблему дрейфа постоянной составляющей сигнала на выходе перемножителей. 

\begin{figure}[H]
    \centering
    \includegraphics[width=1\linewidth]{variant_3_direct_scheme.png}
    \caption{Расширенная схема приёмника прямого преобразования}
    \label{fig:enter-label}
\end{figure}


\section{Расчёт}
\subsection{Заданные параметры}
\begin{itemize}
    \item Мощность полезного сигнала: $P_{DPCH} = -121$ дБм $= -151$ дБВт -  уровень полезного сигнала (канала DPCH) на входе антенны \newline
    Перевод: дБВт= дБм - 30,  т. е. 0 дБм=1 мВт, то 0 дБВт=1 Вт
    \item Коэффициент усиления МШУ: $G_{LNA} = 17$ дБ (50 раз по мощности)
    \item Коэффициент шума (Noise Figure) МШУ: $NF_{LNA} = 1$ дБ
    \item Затухание в полосовом фильтре (Band-Pass Filter): $L_{BPF} = 2.5$ дБ
    \item Запас на неточности реализации цифровой обработки  (Implementation Margin): $L_{IM} = 2$ дБ
    Учитывает все реальные потери, погрешности, допуски, квантизацию, фазовые шумы и т. д.
    \item Коэффициент шума демодулятора: $NF_{DEM} = 11.6$ дБ
    \item Коэффициент усиления демодулятора: $G_{DEM} = 1.5$ дБ
    \item Затухание ANT-RX: $L_{RX} = 2$ дБ  \newline
    Затухание в дуплексере между выводами "ANT - RX" (между антенной и приёмником) в диапазоне частот приёма. То есть сигнал на входе МШУ будет на 2 дБ слабее, чем на антенне.
    \item Затухание TX-RX: $L_{TX/RX} = 35$ дБ  \newline
    Затухание в дуплексере между выводами "TX - RX" (от передатчика к приёмнику)  в диапазоне частот приёма
    \item Полоса: $BW = 3.84$ МГц
    \item Постоянная Больцмана: $k = 1.38\cdot10^{-23}$ Вт/Гц·К
    \item Температура (комнатная): $T = 273$ К
\end{itemize}
Технические характеристики подобранных элементов описаны в Приложении 1

\subsection{Расчёт чувствительности приёмника}
Реальная чувствительность приёмника определяется как минимальный уровень мощности полезного сигнала PDPCH на входе антенны BS, при котором выходная вероятность ошибки на бит BER стандартного измерительного канала передачи данных не превышает заданной величины.
\subsubsection{Исходные данные}
Выпишем и посчитаем данные, которые понадобятся для этого пункта.
\begin{itemize}
\item Мощность полезного сигнала: $P_{DPCH} = -121$ дБм
\item Системный шум от базовой станции:
\begin{equation}
P_{N}^{SYS} = 10 · log_{10}(10^{P_{RX}/10} - 10^{P_{DPCH}/10})= -111.1 \text{дБм}
\end{equation}
\item Шум от передатчика (через дуплексер): $P_{NTX} = -70 - L_{TX/RX} = -105$ дБм
\item Коэффициент шума МШУ: $NF_{LNA} = 1$ дБ
\item Усиление МШУ: $G_{LNA} = 17$ дБ
\item Затухание в полосовом фильтре: $L_{BPF} = 2.5$ дБ
\item Коэффициент шума демодулятора: $NF_{DEM} = 11.6$ дБ
\item Усиление демодулятора: $G_{DEM} = 1.5$ дБ
\item Ширина полосы: $BW = 3.84$ МГц 
\item Постоянная Больцмана: $k = 1.38\cdot10^{-23}$ Вт/Гц·К
\item Температура: $T = 273$ К 
\item Implementation Margin: запас на реализацию цифровой обработки:\\ $L_{IM} = 2$ дБ
\item Processing Gain: энергетический выигрыш вследствие свёртки шумоподобного сигнала PNS (определяется из соотношения ширины спектра сигнала WCDMA и полосы полезного информационного сигнала после свёртки): 
\begin{equation}
G_{PG} = 10  log_{10}\frac{BW}{R_{data}} = 10log_{10}(314.75) = 25 \text{дБ}
\end{equation}
\end{itemize}
\subsubsection{Расчёт теплового шума и суммарного коэффициента шума}
Кроме системных имеется ещё два источника помех:  \\
1) аддитивный белый гауссовский шум (Additive White Gaussian Noise - AWGN), обусловленный собственными тепловыми шумами каскадов усиления приемника (особенно его входного LNA);\\
2) шумовая составляющая шумов передатчика BS в диапазоне принимаемых частот, спектральную плотность которого можно также считать постоянной. Рассчитывается с учетом затухания сигнала в дуплексере и мощности собственных шумов передатчика в диапазоне частот приёма PN(BW) = -70 дБм -- эта величина была получена на основе анализа шумов типовых передатчиков данного диапазона\\

Тепловой шум в полосе $BW$:
\begin{equation}
P_{thermal} = 10 \cdot \log_{10}(kTB) + 30 = 10 \cdot \log_{10}(1.38 \cdot 10^{-23} \cdot 273 \cdot 3.84 \cdot 10^6) + 30 \approx -108.4\ \text{дБм}
\end{equation}

Суммарный коэффициент шума приёмного тракта:
\begin{equation}
NF_\Sigma = 10 \cdot \log_{10} \left(10^{\frac{NF_{LNA}}{10}} + \frac{10^{\frac{NF_{DEM}}{10}} - 1}{10^{\frac{G_{LNA} - L_{BPF}}{10}}} \right) 
\end{equation}

\begin{align}
NF_\Sigma &= 10 \cdot \log_{10} \left(10^{\frac{1}{10}} + \frac{10^{\frac{11.6}{10}} - 1}{10^{\frac{17 - 2.5}{10}}} \right) \notag \\
&= 10 \cdot \log_{10} \left(1.26 + \frac{14.45 - 1}{28.18} \right) \notag \\
&= 10 \cdot \log_{10} (1.26 + 0.477) = 10 \cdot \log_{10} (1.737) \approx 2.4\ \text{дБ}
\end{align}
\\
Суммарный шум приёмника:
\begin{equation}
P_{N0} = P_{thermal} + NF_\Sigma = -108.4\text{дБм} + 2.4 \text{дБм} = -106\ \text{дБм}
\end{equation}

\subsubsection{Расчёт результирующего отношения $S / (N + I)$}
Суммарная помеха:
\begin{equation}
P_{NI} = 10 \cdot \log_{10}\left(10^{\frac{P_{N}^{SYS}}{10}} + 10^{\frac{P_{N0}}{10}} + 10^{\frac{P_{NTX}}{10}}\right)
\end{equation}
Подставляя значения:
\begin{equation}
P_{NI} = 10 \cdot \log_{10}\left(10^{-11.11} + 10^{-10.60} + 10^{-10.5}\right) \approx -104.9\ \text{дБм}
\end{equation}

Отношение сигнал/шум+помеха:
\begin{equation}
\left(\frac{S}{N+I}\right) = P_{DPCH} - P_{NI} = -121 - (-104.9) = -16.1\ \text{дБ}
\end{equation}

\subsubsection{Переход к эффективному значению}
\begin{equation}
\left(\frac{S}{N+I}\right)_{EFF} = \left(\frac{S}{N+I}\right) - L{IM} + G_{PG} = -16.1 - 2 + 25 = 6.9\ \text{дБ}
\end{equation}

\subsubsection{Вывод:} полученное значение $6.9$ дБ превышает требуемое значение $5.2$ дБ с заметным запасом, что подтверждает выполнение условия чувствительности приёмника.

\subsection{Расчёт избирательности по соседнему каналу}
\subsubsection{Исходные данные}
\begin{itemize}
\item Уровень полезного сигнала: $P_{wanted} = -115$ дБм
\item Уровень мешающего сигнала: $P_{adj} = -52$ дБм
\item Смещение по частоте: $\pm 5$ МГц
\item Требуемое значение $E_b/N_0$: 5.2 дБ
\item Processing Gain: $G_{PG} = 25$ дБ
\item Implementation Margin: $L_{IM} = 2$ дБ
\end{itemize}

Мешающий сигнал создаёт дополнительную помеху, которая поступает через фильтр нижних частот (ФНЧ). Чтобы приём был возможен, сигнал после ФНЧ должен удовлетворять требованию:
\begin{equation}
\left( \frac{S}{N + I + I_{adj}} \right)_{eff} \geq 5.2\ \text{дБ}
\end{equation}

\subsubsection{Расчёт допустимого уровня помехи от соседнего канала после ФНЧ}
Рассчитаем итоговое отношение сигнал/шум/помеха без учёта фильтрации:
\begin{equation}
P_{total} = 10 \cdot \log_{10}\left(10^{P_{NSYS}/10} + 10^{P_{N0}/10} + 10^{P_{NTX}/10} + 10^{P_{adj}/10} \right)
\end{equation}

Подставим:
\begin{equation}
P_{total} = 10 \cdot \log_{10}\left(10^{-11.11} + 10^{-11.8} + 10^{-10.5} + 10^{-5.2} \right) \approx -52.0\ \text{дБм}
\end{equation}

\begin{equation}
\left( \frac{S}{N + I + I_{adj}} \right) = P_{wanted} - P_{total} = -115 - (-52) = -63\ \text{дБ}
\end{equation}

Поскольку требуемое значение после учёта запаса и выигрыша:
\begin{equation}
\left( \frac{S}{N + I + I_{adj}} \right){eff} = -63 + G{PG} - L_{IM} = -63 + 25 - 2 = -40\ \text{дБ} \ll 5.2\ \text{дБ}
\end{equation}

Следовательно, необходимо подавить мешающий сигнал фильтром минимум на:
\begin{equation}
A_{FILT} = 5.2 + L_{IM} - G_{PG} + (P_{adj} - P_{wanted}) = 5.2 + 2 - 25 + ( -52 + 115 ) = 45.2\ \text{дБ}
\end{equation}

\subsubsection{Вывод}
Для обеспечения приёма при наличии сильного мешающего сигнала в соседнем канале (–52 дБм при полезном –115 дБм), фильтр нижних частот должен обеспечивать подавление не менее 45.2 дБ на частоте $\pm 5$ МГц. Подходящими являются фильтры Баттерворта 7-го порядка или Чебышёва 6-го порядка.




































\end{document}

