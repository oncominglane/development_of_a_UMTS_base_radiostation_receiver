\documentclass[a4paper,12pt]{article}
\usepackage[utf8]{inputenc}
\usepackage[russian]{babel}
\usepackage{amsmath, amssymb}
\usepackage{graphicx}
\usepackage{float}
\usepackage{geometry}
\geometry{top=2cm,bottom=2.5cm,left=3cm,right=2.5cm}

\usepackage[colorlinks=true, urlcolor=blue, linkcolor=black]{hyperref}

\newpage\begin{document}

\thispagestyle{empty} % без номера страницы
\begin{center}
\large \textbf{Министерство цифрового развития, связи и массовых коммуникаций Российской Федерации}\\[0.8em]
\large Ордена Трудового Красного Знамени федеральное государственное бюджетное образовательное учреждение высшего образования \\[0.3em]
\large \textbf{Московский технический университет связи и информатики}  \\[0.4em]
\large Кафедра «Радиооборудование и схемотехника»\\[1em]
\end{center}

\vspace*{3cm}

\begin{center}
\large \textbf{Курсовой проект} \\[1em]
\large По дисциплине: \\«Радиоприёмные устройства» \\[1em]
\Large \textbf{Разработка приёмника базовой станции UMTS  (диапазон 1850–1910 МГц)} \\[3em]
\end{center}

\vfill % всё, что после этого, прижмётся к низу страницы

\begin{flushright}
Выполнил:   Игров А.М \qquad\qquad \\(гр. БРМ2202) \\
Проверил:   Пестряков А.В.\\[3.5em]
\end{flushright}

\begin{center}
Москва 2025 г.
\end{center}

\newpage
\tableofcontents
\newpage

\section{Техническое задание}
Разработать радиоприёмное устройство базовой станции стандарта UMTS (универсальной мобильной телекоммуникационной системы с прямым расширением спектра), используя техническое задание, представленное в таблице \ref{tab:t1}.

\begin{table}[H]
\centering
\caption{Технические характеристики}
\label{tab:t1}
\begin{tabular}{|l|l|}
\hline
\textbf{Параметр} & \textbf{Значение} \\ \hline
Принцип дуплексирования & Частотное разделение, разнос 95 МГц \\ \hline
Диапазон приёма & 1850--1910 МГц \\ \hline
Диапазон передачи & 1945--2005 МГц \\ \hline
Модуляция & QPSK \\ \hline
Шаг перестройки по частоте & 5 МГц, дискрет 200 кГц \\ \hline
Точность частоты & $\pm$0.05 ppm (Wide Area BS) \\ \hline
BER & $\leq$ 0.001 \\ \hline
Eb/N0 & $\geq$ 5.2 дБ \\ \hline
Чувствительность BS & --121 дБм (12.2 kbps) \\ \hline
Избирательность по соседнему каналу & --115 дБм / --52 дБм, $\pm$5 МГц \\ \hline
Избирательность по побочным каналам & Согласно 3GPP TS 25.101 \\ \hline
Динамический диапазон & --110.7 дБм ... --25 дБм (85.7 дБ) \\ \hline
\end{tabular}
\end{table}

\section{Выбор структуры приёмника}
Для выбора схемы приёмника проведем сравнение возможных вариантов реализации.
\subsection{Супергетеродинный приёмник}
\textbf{Принцип работы:} \newline
ПФ1 совместно с фильтром ПФ2 ослабляет уровень помех по зеркальному и другим побочным каналам. МШУ обеспечивает заданную чувствительность. Побочные продукты преобразования подавляются ФПЧ1.
\newline
\newline
\textbf{Преимущества:}
\begin{itemize}
    \item высокая чувствительность и избирательность,
    \item стабильные параметры.
\end{itemize}
\textbf{Недостатки:}
\begin{itemize}
    \item большое количество компонентов,
    \item сложность микроминиатюризации,
    \item повышенное энергопотребление.
\end{itemize}

\begin{figure}[H]
    \centering
    \includegraphics[width=1\linewidth]{variant_1_sup.png}
    \caption{Схема супергетеродинного приёмника с двукратным преобразованием частоты}
    \label{fig:enter-label}
\end{figure}

\subsection{Приёмник прямого преобразования}
\textbf{Принцип работы:} \newline
квадратурный преобразователь частоты переносит спектр сигнала на две низкочастотные составляющие. УНЧ и ФНЧ выполняют частотную селекцию.   
\newline
\newline
\textbf{Преимущества:}
\begin{itemize}
    \item простота схемы,
    \item минимум внешних компонентов,
    \item возможность реализации в ИМС,
    \item низкое энергопотребление.
\end{itemize}
\textbf{Недостатки:}
\begin{itemize}
    \item утечка гетеродина → постоянная составляющая,
    \item требования к симметрии I/Q каналов,
    \item интермодуляционные искажения.
\end{itemize}
Появление постоянной составляющей на выходе ФНЧ связано в первую очередь с утечкой сигнала гетеродина. В качестве эффективного решения данной проблемы обычно применяют переход на синтезатор частоты с удвоенной рабочей частотой. Частота, равная частоте входного сигнала, получается уже внутри ИМС путем деления на 2, что приводит к почти полному исчезновению излучения через паразитные цепи. Также, правильная компоновка компонентов РЧ блоков, экранирование узлов и применение специальных алгоритмов оценивания в цифровом блоке обработки, помогают устранить большинство недостатков этой схемы.

\begin{figure}[H]
    \centering
    \includegraphics[width=1\linewidth]{variant_2_zero_if.png}
    \caption{Схема приёмника с прямым преобразованием}
    \label{fig:enter-label}
\end{figure}

\subsection{Приёмник с цифровой обработкой на ПЧ}
\textbf{Принцип работы:}\newline
спектр сигнала переносится на ПЧ, затем оцифровывается АЦП и демодулируется цифровыми средствами (DDS, цифровые ФНЧ, I/Q).
\newline
\newline
\textbf{Преимущества:}
\begin{itemize}
    \item идеальная симметрия каналов,
    \item программная гибкость,
    \item многоканальная обработка.
\end{itemize}
\textbf{Недостатки:}
\begin{itemize}
    \item необходимость быстрого АЦП,
    \item повышенное энергопотребление,
    \item сложность и стоимость реализации.
\end{itemize}
\begin{figure}[H]
    \centering
    \includegraphics[width=1\linewidth]{variant_3_digital_if.png}
    \caption{Схема приёмника с цифровой обработкой на ПЧ}
    \label{fig:enter-label}
\end{figure}



\subsection{Выбор схемы}
Исходя из указанных плюсов и минусов структур приёмника была выбрана схема приёмника прямого преобразования. Расширенная схема такого приёмника на рис. 3. За счёт малой элементной базы структура приемника прямого преобразования будет более простой для реализации, а большинство недостатков удастся избежать правильно подобранной элементной базой, различным экранированием, использованием дифференциальных схем гетеродинов и смесителей, а также применением схем (алгоритмов) оценки и компенсации дрейфа постоянной составляющей и не идентичности каналов. За счёт чего удаётся существенно ослабить проблему дрейфа постоянной составляющей сигнала на выходе перемножителей. 

\begin{figure}[H]
    \centering
    \includegraphics[width=1\linewidth]{variant_3_direct_scheme.png}
    \caption{Расширенная схема приёмника прямого преобразования}
    \label{fig:enter-label}
\end{figure}







\newpage

\section{Выбор компонентов системы}

\subsection{Выбор малошумящего усилителя (МШУ)}

Для обеспечения необходимой чувствительности приёмника и минимизации шумов, в качестве малошумящего усилителя выбран чип \textbf{Infineon BGA9H1MN9}. Он обеспечивает высокий коэффициент усиления и низкий коэффициент шума в диапазоне частот UMTS 1850–1910 МГц.

\begin{table}[H]
\centering
\caption{Технические характеристики МШУ BGA9H1MN9}
\begin{tabular}{|l|l|}
\hline
\textbf{Параметр} & \textbf{Значение} \\ \hline
Фирма-изготовитель & Infineon \\ \hline
Модель & BGA9H1MN9 \\ \hline
Диапазон частот & 1700–2200 МГц \\ \hline
Коэффициент усиления & 20 дБ \\ \hline
Коэффициент шума (Noise Figure) & 0.8 дБ \\ \hline
Рабочее напряжение питания & 3.3 В \\ \hline
Ток потребления & 50 мА \\ \hline
Тип корпуса & TSLP-9 \\ \hline
\end{tabular}
\end{table}

\textbf{Вывод:} усилитель BGA9H1MN9 обеспечивает необходимые параметры усиления и шумов в нужном диапазоне частот и подходит для использования в тракте приёмника.

\subsection{Выбор полосового ПАВ-фильтра (SAW)}

Для фильтрации принимаемого сигнала и подавления побочных каналов выбран SAW-фильтр \textbf{SF1880BA02524S}, обладающий подходящей центральной частотой и малыми потерями в полосе пропускания.

\begin{table}[H]
\centering
\caption{Технические характеристики SAW-фильтра SF1880BA02524S}
\begin{tabular}{|l|l|}
\hline
\textbf{Параметр} & \textbf{Значение} \\ \hline
Фирма-изготовитель & Spectrum Control \\ \hline
Модель & SF1880BA02524S \\ \hline
Центральная частота & 1880 МГц \\ \hline
Рабочий диапазон частот & 1850–1910 МГц \\ \hline
Вставные потери (Insertion loss) & 2.45 дБ (тип.) \\ \hline
Пульсации в полосе & 1.2 дБ (тип.) \\ \hline
АЧХ вне полосы (аттенюация) & ≥ 30 дБ вне диапазона \\ \hline
Согласованная нагрузка & 50 Ом \\ \hline
\end{tabular}
\end{table}

\textbf{Вывод:} фильтр SF1880BA02524S обеспечивает малое затухание и хорошую фильтрацию вне полосы приёма в диапазоне 1850–1910 МГц и может использоваться в приёмнике UMTS-базовой станции.



\subsection{Выбор демодулятора}

Для реализации квадратурного приёма и преобразования радиочастотного сигнала в базовую полосу (I/Q), в приёмном тракте используется демодулятор \textbf{Analog Devices ADRF6820}. Он обеспечивает стабильные характеристики в широком диапазоне частот и подходит для диапазона UMTS 1850–1910 МГц.

\begin{table}[H]
\centering
\caption{Технические характеристики демодулятора ADRF6820}
\begin{tabular}{|l|l|}
\hline
\textbf{Параметр} & \textbf{Значение} \\ \hline
Фирма-изготовитель & Analog Devices \\ \hline
Модель & ADRF6820 \\ \hline
Диапазон входных частот (RF) & 695–2700 МГц \\ \hline
Диапазон частот гетеродина (LO) & 356–2850 МГц \\ \hline
Тип выходного сигнала & I/Q, дифференциальный \\ \hline
Коэффициент усиления (Conversion Gain) & 1.5 дБ \\ \hline
Коэффициент шума (Noise Figure) & 11.6 дБ \\ \hline
Ширина полосы демодулированного сигнала & до 600 МГц \\ \hline
Регулируемое ослабление сигнала (DSA) & 0…15 дБ, шаг 1 дБ \\ \hline
Корпус & LFCSP-40, 6×6 мм \\ \hline
\end{tabular}
\end{table}

\textbf{Вывод:} демодулятор ADRF6820 подходит по диапазону рабочих частот и обеспечивает необходимые характеристики усиления и фильтрации для корректной работы приёмного тракта в UMTS 1850–1910 МГц.

\subsection{Выбор дуплексера}

Дуплексер используется для раздельного приёма и передачи сигналов в одном антенно-фидерном тракте. Он обеспечивает разделение частотных каналов передачи и приёма, а также необходимую изоляцию между ними. В соответствии с техническим заданием, система работает в диапазоне частот приёма 1850–1910 МГц и передачи 1945–2005 МГц, что требует дуплексера с разносом 95 МГц.

В связи с отсутствием на рынке стандартных дуплексеров на диапазон (1850–1910 / 1945–2005 МГц), выбран ближайший по параметрам компонент — \textbf{AM1880-1960D1004}. Он обеспечивает необходимые характеристики в диапазоне приёма 1850–1910 МГц, но диапазон передачи ограничен верхней границей 1990 МГц, в связи с чем возможна потеря до 15 МГц верхней части диапазона передачи. Несмотря на это, дуплексер обеспечивает высокую изоляцию и низкие потери, что делает его пригодным для демонстрационной и лабораторной реализации устройства.

\begin{table}[H]
\centering
\caption{Технические характеристики дуплексера AM1880-1960D1004}
\begin{tabular}{|l|l|}
\hline
\textbf{Параметр} & \textbf{Значение} \\ \hline
Фирма-изготовитель & Anatech Electronics \\ \hline
Модель & AM1880-1960D1004 \\ \hline
Диапазон частот приёма (RX) & 1850–1910 МГц \\ \hline
Диапазон частот передачи (TX) & 1930–1990 МГц \\ \hline
Изоляция RX/TX & ≥ 60 дБ \\ \hline
Затухание (Insertion Loss) & 2 дБ \\ \hline
Return Loss & 18 дБ \\ \hline
Импеданс & 50 Ом \\ \hline
Мощность & 5 Вт \\ \hline
\end{tabular}
\end{table}

\textbf{Вывод:} несмотря на неполное перекрытие диапазона передачи, дуплексер AM1880-1960D1004 обеспечивает необходимые характеристики по приёму и хорошую изоляцию. Он может использоваться в данном проекте как технически обоснованная альтернатива в условиях ограниченного ассортимента компонентов.

Стоит заметить, что при расчётах принято типовое значение $L_{TX/RX} = 35$ дБ, несмотря на то, что фактическая изоляция дуплексера AM1880-1960D1004 превышает 60 дБ.


\subsection{Выбор аналого-цифрового преобразователя (АЦП)}

Для преобразования сигнала в цифровую форму после усиления и фильтрации выбран двухканальный АЦП \textbf{Analog Devices AD6600}. Он обеспечивает необходимую полосу, частоту дискретизации и динамический диапазон, а также поддерживает автоматическую регулировку усиления (AGC), что особенно важно при переменном уровне входного сигнала.

\begin{table}[H]
\centering
\caption{Технические характеристики АЦП AD6600}
\begin{tabular}{|l|l|}
\hline
\textbf{Параметр} & \textbf{Значение} \\ \hline
Фирма-изготовитель & Analog Devices \\ \hline
Модель & AD6600 \\ \hline
Диапазон входных частот & 70–250 МГц \\ \hline
Разрядность & 11 бит \\ \hline
Частота дискретизации & до 20 Мвыб/с на канал \\ \hline
Количество каналов & 2 (раздельных или diversity) \\ \hline
Полный динамический диапазон & >100 дБ (с AGC и цифровой обработкой) \\ \hline
Встроенное автоматическое\\ регулирование усиления & до 30 дБ \\ \hline
Питание & 5 В \\ \hline
Потребляемая мощность & 775 мВт \\ \hline
Тип корпуса & 44-контактный LQFP \\ \hline
\end{tabular}
\end{table}

\textbf{Вывод:} АЦП AD6600 полностью соответствует требованиям системы по полосе, точности и скорости преобразования. Встроенная функция AGC упрощает согласование уровней сигнала и повышает стабильность работы тракта при переменных условиях приёма.


\subsection{Выбор программируемого усилителя с ФНЧ}

Для стабилизации уровня сигнала перед подачей на АЦП, а также подавления мешающих сигналов за пределами полезной полосы, в видеотракте приёмника используется программируемый усилитель с фильтром нижних частот \textbf{LTC6603}. Он идеально подходит для работы после демодуляции, обеспечивая линейно-фазовую фильтрацию и точную настройку усиления в диапазоне до 24 дБ.

\begin{table}[H]
\centering
\caption{Технические характеристики LTC6603}
\begin{tabular}{|l|l|}
\hline
\textbf{Параметр} & \textbf{Значение} \\ \hline
Фирма-изготовитель & Linear Technology \\ \hline
Модель & LTC6603 \\ \hline
Назначение & ФНЧ + программируемый усилитель \\ \hline
Макс. полоса пропускания & до 2.5 МГц \\ \hline
Программируемое усиление & 0 / 6 / 12 / 24 дБ \\ \hline
Коэффициент шума & --145 дБм/Гц \\ \hline
Фазовая симметрия каналов & 1.5° макс. \\ \hline
Напряжение питания & 2.7–3.6 В \\ \hline
Тип корпуса & QFN 4×4 мм \\ \hline
\end{tabular}
\end{table}

\textbf{Вывод:} компонент LTC6603 отлично подходит для работы в видеотракте после демодуляции сигнала WCDMA. Он обеспечивает необходимое усиление, избирательность и стабильность параметров для дальнейшей цифровой обработки.



\subsection{Выбор усилителя с переменным усилением (VGA)}

Для согласования уровней сигнала и обеспечения необходимой гибкости регулировки усиления в трактах нижних частот выбран чип \textbf{Analog Devices AD8338}. Он обладает широким диапазоном усиления, низким уровнем шума и полностью дифференциальной архитектурой, что делает его идеальным выбором для систем приёма сигналов с низкой частотой (IF).

\begin{table}[H]
\centering
\caption{Технические характеристики усилителя AD8338}
\begin{tabular}{|l|l|}
\hline
\textbf{Параметр} & \textbf{Значение} \\ \hline
Фирма-изготовитель & Analog Devices \\ \hline
Модель & AD8338 \\ \hline
Диапазон частот & До 18 МГц \\ \hline
Диапазон усиления & От 0 до 80 дБ (управляется напряжением) \\ \hline
Шум (Input Referred Noise) & 4.5 нВ/√Гц при усилении 80 дБ \\ \hline
Напряжение питания & 3.0–5.0 В \\ \hline
Потребляемый ток & 3 мА при усилении 40 дБ \\ \hline
Тип корпуса & 16-lead LFCSP \\ \hline
\end{tabular}
\end{table}

\textbf{Вывод:} усилитель AD8338 обеспечивает стабильную работу в пределах необходимой полосы частот и предоставляет гибкие возможности регулировки усиления. Его характеристики соответствуют требованиям приёмного тракта.


























\newpage
\section{Расчёт}
\subsection{Заданные параметры}
\begin{itemize}
    \item Мощность полезного сигнала: $P_{DPCH} = -121$ дБм $= -151$ дБВт -  уровень полезного сигнала (канала DPCH) на входе антенны \newline
    Перевод: дБВт= дБм - 30,  т. е. 0 дБм=1 мВт, то 0 дБВт=1 Вт
    \item Коэффициент усиления МШУ: $G_{LNA} = 20$ дБ 
    \item Коэффициент шума (Noise Figure) МШУ: $NF_{LNA} = 0.8$ дБ
    \item Затухание в полосовом фильтре (Band-Pass Filter): $L_{BPF} = 2.5$ дБ
    \item Запас на неточности реализации цифровой обработки  (Implementation Margin): $L_{IM} = 2$ дБ
    Учитывает все реальные потери, погрешности, допуски, квантизацию, фазовые шумы и т. д.
    \item Коэффициент шума демодулятора: $NF_{DEM} = 11.6$ дБ
    \item Коэффициент усиления демодулятора: $G_{DEM} = 1.5$ дБ
    \item Затухание ANT-RX: $L_{RX} = 2$ дБ  \newline
    Затухание в дуплексере между выводами "ANT - RX" (между антенной и приёмником) в диапазоне частот приёма. То есть сигнал на входе МШУ будет на 2 дБ слабее, чем на антенне.
    \item Затухание TX-RX: $L_{TX/RX} = 60$ дБ  \newline
    Затухание в дуплексере между выводами "TX - RX" (от передатчика к приёмнику)  в диапазоне частот приёма
    \item Полоса: $BW = 3.84$ МГц
    \item Постоянная Больцмана: $k = 1.38\cdot10^{-23}$ Вт/Гц·К
    \item Температура (комнатная): $T = 273$ К
\end{itemize}
Технические характеристики подобранных элементов описаны в Приложении












\subsection{Расчёт чувствительности приёмника}
Реальная чувствительность приёмника определяется как минимальный уровень мощности полезного сигнала PDPCH на входе антенны BS, при котором выходная вероятность ошибки на бит BER стандартного измерительного канала передачи данных не превышает заданной величины.
\subsubsection{Исходные данные}
Выпишем и посчитаем данные, которые понадобятся для этого пункта.
\begin{itemize}
\item Мощность полезного сигнала: $P_{DPCH} = -121$ дБм
\item Системный шум от базовой станции:
\begin{equation}
P_{N}^{SYS} = 10 · log_{10}(10^{P_{RX}/10} - 10^{P_{DPCH}/10})= -110.4 \text{дБм}
\end{equation}
\item Шум от передатчика (через дуплексер): $P_{N}^{TX} = -70 - L_{TX/RX} = -130$ дБм
\item Коэффициент шума МШУ: $NF_{LNA} = 0.8$ дБ
\item Усиление МШУ: $G_{LNA} = 20$ дБ
\item Затухание в полосовом фильтре: $L_{BPF} = 2.45$ дБ
\item Коэффициент шума демодулятора: $NF_{DEM} = 11.6$ дБ
\item Усиление демодулятора: $G_{DEM} = 1.5$ дБ
\item Ширина полосы: $BW = 3.84$ МГц 
\item Постоянная Больцмана: $k = 1.38\cdot10^{-23}$ Вт/Гц·К
\item Температура: $T = 273$ К 
\item Implementation Margin: запас на реализацию цифровой обработки:\\ $L_{IM} = 2$ дБ
\item Processing Gain: энергетический выигрыш вследствие свёртки шумоподобного сигнала PNS (определяется из соотношения ширины спектра сигнала WCDMA и полосы полезного информационного сигнала после свёртки): 
\begin{equation}
G_{PG} = 10  log_{10}\frac{BW}{R_{data}} = 10log_{10}(314.75) = 25 \text{дБ}
\end{equation}
\end{itemize}
\subsubsection{Расчёт теплового шума и суммарного коэффициента шума}
Кроме системных имеется ещё два источника помех:  \\
1) аддитивный белый гауссовский шум (Additive White Gaussian Noise - AWGN), обусловленный собственными тепловыми шумами каскадов усиления приемника (особенно его входного LNA);\\
2) шумовая составляющая шумов передатчика BS в диапазоне принимаемых частот, спектральную плотность которого можно также считать постоянной. Рассчитывается с учетом затухания сигнала в дуплексере и мощности собственных шумов передатчика в диапазоне частот приёма $P_N(BW) = -70$ дБм -- эта величина была получена на основе анализа шумов типовых передатчиков данного диапазона\\




Тепловой шум в полосе $BW$:
\begin{equation}
P_{thermal} = 10 \cdot \log_{10}(kTB) + 30 = 10 \cdot \log_{10}(1.38 \cdot 10^{-23} \cdot 273 \cdot 3.84 \cdot 10^6) + 30 \approx -108.4\ \text{дБм}
\end{equation}

Суммарный коэффициент шума приёмного тракта:
\begin{equation}
NF_\Sigma = 10 \cdot \log_{10} \left(10^{\frac{NF_{LNA}}{10}} + \frac{10^{\frac{NF_{DEM}}{10}} - 1}{10^{\frac{G_{LNA} - L_{BPF}}{10}}} \right)
\end{equation}
Подставляем значения:
\begin{align}
NF_\Sigma = 10 \cdot \log_{10} \left(10^{\frac{0.8}{10}} + \frac{10^{\frac{11.6}{10}} - 1}{10^{\frac{20 - 2.45}{10}}} \right)= 10 \cdot \log_{10} \left(1.202 + \frac{14.45 - 1}{56.96} \right) \notag \\
= 10 \cdot \log_{10} (1.202 + 0.237) = 10 \cdot \log_{10} (1.439) \approx 1.58\ \text{дБ}
\end{align}
\\
Суммарный шум приёмника:
\begin{equation}
P_{N0} = P_{thermal} + NF_\Sigma = -108.4\ \text{дБм} + 1.58\ \text{дБ} = -106.82\ \text{дБм}
\end{equation}

\subsubsection{Расчёт результирующего отношения «сигнал/(шум + помеха)»}
Суммарная помеха:
\begin{equation}
P_{NI} = 10 \cdot \log_{10}\left(10^{\frac{P_N^{SYS}}{10}} + 10^{\frac{P_N^0}{10}} + 10^{\frac{P_N^{TX}}{10}}\right)
\end{equation}
Подставим:
\begin{equation}
P_{NI} = 10 \cdot \log_{10}\left(10^{\frac{-110.36}{10}} + 10^{\frac{-106.82}{10}} + 10^{\frac{-130}{10}}\right) \approx -105.21\ \text{дБм}
\end{equation}

Отношение сигнал/шум+помеха:
\begin{equation}
\left(\frac{S}{N+I}\right) = P_{DPCH} - P_{NI} = -121 - (-105.21) = -15.79\ \text{дБ}
\end{equation}

\subsubsection{Переход к эффективному значению}
\begin{equation}
\left(\frac{S}{N+I}\right)_{EFF} = \left(\frac{S}{N+I}\right) - L_{IM} + G_{PG} = -15.79 - 2 + 25 = 7.21\ \text{дБ}
\end{equation}



\subsubsection{Вывод:} полученное значение $7.2$ дБ заметно превышает требуемое значение в $5.2$ дБ, что демонстрирует уверенное выполнение условия условия чувствительности приёмника.
\\




















\subsection{Расчёт избирательности по соседнему каналу}
Избирательность по соседнему каналу (ACS – Adjacent Channel Selectivity) является мерой способности приёмника принимать полезный WCDMA сигнал с заданным уровнем качества (величина BER не превышает 10-3) в присутствии мешающего сигнала по соседнему каналу (смещение по частоте на ±5 МГц).
\subsubsection{Исходные данные}
\begin{itemize}
\item Мощность полезного сигнала: $P_{DPCH} = -115$ дБм
\item Уровень мешающего сигнала: $P_{adj} = -52$ дБм
\item Системный шум: $P_{N}^{SYS} = -110.36$ дБм
\item Шум приёмника: $P_{N}^{0} = -106.82$ дБм
\item Шум от передатчика: $P_{N}^{TX} = -130$ дБм
\item Смещение по частоте: $\pm 5$ МГц
\item Требуемое значение $E_b/N_0$: 5.2 дБ
\item Processing Gain: $G_{PG} = 25$ дБ
\item Implementation Margin: $L_{IM} = 2$ дБ
\end{itemize}

Мешающий сигнал создаёт дополнительную помеху, которая поступает через фильтр нижних частот (ФНЧ). Чтобы приём был возможен, сигнал после ФНЧ должен удовлетворять требованию:
\begin{equation}
\left( \frac{S}{N + I + I_{adj}} \right)_{EFF} \geq 5.2\ \text{дБ}
\end{equation}


\subsubsection{Расчёт отношения «сигнал/(шум+помеха)»}
\begin{equation}
\frac{S}{N + I} = P_{DPCH} - 10 \cdot \log_{10} \left(10^{P_{N}^{SYS}/10} + 10^{P_{N}^{0}/10} + 10^{P_{N}^{TX}/10} \right)
\end{equation}

Подставим значения:
\begin{equation}
\frac{S}{N + I} = -115 - 10 \cdot \log_{10} \left(10^{-11.036} + 10^{-10.682} + 10^{-13.0} \right) \approx -9.79\ \text{дБ}
\end{equation}

Эффективное значение:
\begin{equation}
\left( \frac{S}{N + I} \right)_{EFF} = \left( \frac{S}{N + I} \right) + G_{PG} - L_{IM} = -9.79 + 25 - 2 = 13.21\ \text{дБ}
\end{equation}



\subsubsection{Расчёт отношения «сигнал/помеха» без фильтрации}
\begin{equation}
\left( \frac{S}{I_{ACI}} \right)_{EFF} = P_{DPCH} - P_{ACI} + G_{PG} - L_{IM} = -115 + 52 + 25 - 2 = -40\ \text{дБ}
\end{equation}
Так как нужно добиться показателя $>E_B/N_0 =5.2$ дБ, т. е.  минимального значения эффективного отношения «сигнал/помеха», для обеспечения приёма требуется:
\begin{equation}
\left( \frac{S}{I_{ACI}} \right)_{EFF}^{target} = 6\ \text{дБ}
\end{equation}

Тогда необходимое подавление фильтром:
\begin{equation}
L_{LPF}(\Delta f) = 6 - (-40) = 46\ \text{дБ}
\end{equation}

\textbf{Вывод:} фильтр должен ослаблять мешающий сигнал на частоте $\pm5$ МГц минимум на 46 дБ, чтобы обеспечить требуемое качество приёма.  

\subsubsection{Обоснование выбора цифрового фильтра}
Данную избирательность мы с запасом можем реализовать при помощи микросхемы \textbf{LTC6603}: согласно даташиту «LTC6603 - Dual Adjustable Lowpass Filter», фильтр обеспечивает подавление сигнала в диапазоне ±5 МГц выше 46 дБ при правильной настройке.

Для приёма WCDMA-сигнала (UMTS) с частотой чипирования 3.84 МГц, полоса после демодуляции составляет 1.92 МГц. Это и есть требуемая частота среза фильтра:

\begin{equation}
f_{\text{cutoff}} = \frac{3.84}{2} = 1.92\ \text{МГц}
\end{equation}

Настроив LTC6603 на частоту среза 1.92 МГц, получаем следующее:
\begin{itemize}
    \item Полезный сигнал (до 1.92 МГц) проходит без искажений.
    \item Мешающий сигнал находится на ±5 МГц, то есть:
    \[
    \frac{f_{\text{помехи}}}{f_{\text{среза}}} = \frac{5}{1.92} \approx 2.6
    \]
    \item По графику из даташита видно, что подавление на такой частоте превышает 50 дБ.
\end{itemize}




\begin{figure}[H]
    \centering
    \includegraphics[width=1\linewidth]{filter.png}
    \caption{Характеристики чипа LTC6603}
    \label{fig:enter-label}
\end{figure}

Слева на рис.5 — амплитудно-частотная характеристика (АЧХ) фильтра:
\begin{itemize}
\item В области ниже 1.92 МГц сигнал практически не ослабляется (усиление ≈ 0 дБ).
\item На частоте ~5 МГц подавление составляет ≈50–55 дБ, в зависимости от конфигурации.
\item При частотах >6 МГц уровень подавления превышает 60 дБ.
\end{itemize}

Справа на графике — групповая задержка:
\begin{itemize}
\item Групповая задержка стабильна в пределах полосы пропускания.
\item Это важно для систем с цифровой обработкой, чтобы избежать фазовых искажений.
\end{itemize}

\textbf{Вывод}: При частоте среза 1.92 МГц LTC6603 обеспечивает стабильное подавление мешающего сигнала на уровне $\geq 50$ дБ, что перекрывает необходимое значение 46 дБ и гарантирует выполнение требований по избирательности по соседнему каналу.


\subsubsection{Расчёт итогового эффективного отношения «сигнал/(шум + помеха)»}

Рассчитаем новое значение эффективного отношения «сигнал/(шум + помеха)» с учётом подавления ФНЧ в выбранной ИМС, $L_{LPF}(\Delta F) = 60$ дБ:

\begin{equation}
\left( \frac{S}{I_{ACI}} \right)_{EFF} = L_{LPF}(\Delta F) - 40 = 60 - 40 = 20\ \text{дБ}
\end{equation}

\begin{align}
\left( \frac{S}{N + I_{ACI}} \right)_{EFF} = 10 \cdot \log_{10} \left( \frac{1}{10^{-\left( \frac{S}{N+I} \right)_{EFF}/10} + 10^{-\left( \frac{S}{I_{ACI}} \right)_{EFF}/10}} \right) \notag \\=
10 \cdot \log_{10} \left( \frac{1}{10^{-13.21/10} + 10^{-20/10}} \right) = 10 \cdot \log_{10} (17.32)
\approx 12.38\ \text{дБ}
\end{align}

\subsubsection{Вывод}
Итоговое эффективное отношение $\left( \frac{S}{N + I_{ACI}} \right)_{EFF} = 12.38$ дБ $> E_b/N_0 = 5.2$ дБ.
Устройство обеспечивает качественный приём сигнала даже при наличии мешающих сигналов в соседнем канале, с большим запасом.
Полученный запас помогает бороться со многими дополнительными факторами (помехами), которые не учитывались в расчёте.

















\subsection{Требуемое сквозное усиление приёмника}
\subsubsection{Диапазон входных сигналов}
Согласно стандарту UMTS (3GPP TS 25.101), минимальная мощность входного сигнала, при котором приём должен быть обеспечен — $P_{min} = -115$ дБм. 

Максимальная входная мощность, не вызывающая перегрузку тракта, принимается как $P_{max} = -25$ дБм.

\begin{equation}
D = P_{max} - P_{min} = -25 - (-115) = 90\ \text{дБ}
\end{equation}

Таким образом, требуемый динамический диапазон приёмника составляет 90 дБ.

\subsubsection{Выбор АЦП и допустимые уровни}
Для цифровой обработки используется АЦП с разрядностью 11 бит. Типовое значение эффективного динамического диапазона такого АЦП составляет около 66-72 дБ.

Примем, что сигнал на входе АЦП должен находиться в диапазоне от:
\begin{itemize}
\item $P_{out_min} = -12$ дБм — минимальный уровень, необходимый для устойчивой работы АЦП
\item $P_{out_max} = -2$ дБм — максимальный уровень, не вызывающий искажений
\end{itemize}

\subsubsection{Оценка минимального и максимального усиления тракта}

Для обеспечения требуемого уровня сигнала на выходе, усиление приёмного тракта должно быть в пределах:

\begin{equation}
G_{min} = P_{out\_min} - P_{max} = -12 - (-25) = 13\ \text{дБ}
\end{equation}

\begin{equation}
G_{max} = P_{out\_max} - P_{min} = -2 - (-115) = 113\ \text{дБ}
\end{equation}

Таким образом, диапазон регулировки усиления должен составлять:

\begin{equation}
D = G_{max} - G_{min} = 113 - 13 = 100\ \text{дБ}
\end{equation}

\subsubsection{Параметры выбранного АЦП}
Для работы в тракте выбран AD6600, имеющий полную шкалу 2 В. Во всём динамическом диапазоне на выходе аналоговой части приемного тракта должен быть обеспечен уровень напряжения, примерно соответствующий половине полной шкалы используемого в Baseband процессоре АЦП, этот запас по номинальному уровню в 6дБ (2 раза) необходим для отсутствия искажений сигнала из-за ограничения изменяющейся амплитудной огибающей.  


\subsubsection{Оценка максимальной мощности на входе АЦП}
АЦП имеет дифференциальный двухканальный вход с уровнем полной шкалы 2 В пик-пик. Для исключения искажений, вызванных амплитудной огибающей сигнала, закладывается запас по уровню 6 дБ. Это соответствует снижению амплитуды в 2 раза, т.е. использование только половины шкалы АЦП.

При этом максимальное выходное напряжение аналогового тракта должно быть:
\begin{equation}
U_{pp} = 1\ \text{В (пик-пик)},\quad U_{peak} = 0.5\ \text{В}
\end{equation}

Эффективное (среднеквадратичное) значение напряжения:
\begin{equation}
    U_{EFF} = \frac{U_{peak}}{\sqrt{2}} = \frac{0.5}{\sqrt{2}} \approx 0.356\ \text{В}
\end{equation}

Мощность сигнала на нагрузке 200 Ом:
\begin{equation}
P = 10 \cdot \log_{10} \left( \frac{U_{eff}^2}{R} \cdot \frac{1}{1~\text{мВт}} \right)
= 10 \cdot \log_{10} \left( \frac{0.356^2}{200 \cdot 0.001} \right) \approx -2\ \text{дБм}
\end{equation}

\textbf{Вывод:} максимальная допустимая мощность сигнала на входе АЦП при учёте запаса составляет около $-2$ дБм. Этот уровень используется в дальнейшем при расчёте максимального усиления тракта.


\subsubsection{Расчёт диапазона усиления приёмного тракта}

\textbf{Оценка требуемого диапазона усиления}

На основании ранее рассчитанных мощностей:

\begin{itemize}
\item Минимальный уровень сигнала: $P_{min} = -115$ дБм
\item Максимальный уровень сигнала: $P_{max} = -25$ дБм
\item Максимальный выходной уровень (до АЦП): $P_{out_{max}} = -2$ дБм
\item Минимальный выходной уровень (до АЦП): $P_{out_{min}} = -1.98$ дБм
\end{itemize}

\begin{equation}
G_{max} = P_{out_{max}} - P_{min} = -2 - (-115) = 113\ \text{дБ}
\end{equation}

\begin{equation}
G_{min} = P_{out_{min}} - P_{max} = -1.98 - (-25) = 23.02\ \text{дБ}
\end{equation}

\begin{equation}
D = G_{min} \ldots G_{max} = 26.98 \ldots 113\ \text{дБ}
\end{equation}

\textbf{Деление тракта на части}

Приёмный тракт делится на:
\begin{itemize}
\item \textbf{Радиотракт} — от антенны до выхода I/Q демодулятора
\item \textbf{Видеотракт} — от выхода демодулятора до входа АЦП
\end{itemize}

\subsubsection{Расчёт усиления радиотракта}

Учитываются следующие компоненты:
\begin{itemize}
\item Усиление МШУ: $G_{LNA} = 20$ дБ [1]
\item Затухание в дуплексере: $L_{RX} = 2$ дБ [4]
\item Затухание в полосовом фильтре: $L_{BPF} = 2.45$ дБ [2]
\item Усиление I/Q демодулятора: $G_{DEM} = 1.5$ дБ [3]
\end{itemize}

\begin{equation}
K_{rt} = -L_{RX} - L_{BPF} + G_{LNA} + G_{DEM} = -2 - 2.45 + 20 + 1.5 = 17.05\ \text{дБ}
\end{equation}

\textbf{Добавление усиления видеотракта} 
В видеотракт включён УНЧ LTC6603 с фиксированным усилением до 24 дБ со встроенным фильтром нижних частот (ФНЧ).
\begin{equation}
K_\Sigma = K_{рт} + K_{VGA} = 17.05 + 24 = 41.05\ \text{дБ}
\end{equation}
 

\textbf{Проблема}

Требуется до 113 дБ усиления, а получено только 41.05 дБ. Недостаёт около 72 дБ.

\textbf{Решение }

Для компенсации недостающего усиления применяется дополнительный УНЧ AD8338 с переменным усилением 0–80 дБ (не менее 76дБ). Он хорошо подходит для реализации АРУ (автоматической регулировки усиления).



\textbf{Вывод:} 

Один УНЧ (LTC6603) обеспечивает базовое усиление и фильтрацию, а AD8338 — гибкую регулировку. В сумме они обеспечивают диапазон усиления $D = 26.98 \ldots 113$ дБ, что покрывает весь динамический диапазон приёмника.
Также стоит заметить, что выбранные компоненты не зависят от диапазона работы радиостанции, т.к. работают на низких частотах после преобразования частоты и не чувствительны к радиодиапазону.



\subsubsection{Обоснование выбора схемы усиления из двух УНЧ}

В структуре приёмного тракта используются два усилительных каскада: основной УНЧ со встроенным фильтром нижних частот (LTC6603) и дополнительный усилитель с переменным усилением (AD8338). Такой подход обусловлен следующими факторами:

\begin{itemize}
\item \textbf{Функциональное разделение каскадов.} Усилитель LTC6603 выполняет не только функцию усиления сигнала, но и обеспечивает необходимую избирательность по соседнему каналу за счёт встроенного линейно-фазового ФНЧ. AD8338 применяется в качестве регулируемого усилителя (VGA), позволяющего адаптировать усиление под текущий уровень входного сигнала.

\item \textbf{Оптимизация шумовых характеристик.} LTC6603 имеет низкий коэффициент шума даже при усилении до 24 дБ, что позволяет предварительно усилить сигнал без существенного увеличения шумов. Это особенно важно на начальных этапах тракта, где уровень сигнала минимален.

\item \textbf{Гибкость и регулировка.} AD8338 позволяет плавно изменять усиление в диапазоне от 0 до 80 дБ с шагом 0.5 дБ. Это делает возможным реализацию автоматической регулировки усиления (АРУ), обеспечивая стабильную амплитуду сигнала на входе АЦП независимо от уровня принимаемого сигнала.

\item \textbf{Минимизация искажений.} Разделение усиления между двумя каскадами позволяет каждому из них работать в своей линейной области, уменьшая вероятность перегрузки и нелинейных искажений.

\item \textbf{Снижение требований к одному компоненту.} Использование двух усилителей позволяет избежать необходимости в одном усилителе с чрезмерно широким диапазоном регулировки, что упростило выбор компонентов и повысило надёжность схемы.

\end{itemize}

Таким образом, применение двухкаскадной схемы усиления обеспечивает требуемый динамический диапазон усиления приёмного тракта, улучшает шумовые характеристики и повышает стабильность работы устройства в условиях реальных помех и изменений уровня сигнала.





\subsubsection{Окончательное сквозное усиление приёмного тракта}

Учитывая второй каскад усиления (УНЧ2 — AD8338), расчёт полного усиления тракта принимает следующий вид:
\begin{equation}
K_\Sigma = K_{рт} + K_{VGA1} + K_{VGA2} = 17.05 + 24 + 80 = 121.05\ \text{дБ}
\end{equation}

Это значение превышает требуемое максимальное усиление $G_{max} = 113$ дБ и обеспечивает необходимый запас. 

Для обеспечения минимального коэффициента передачи $G_{min} = 26.98$ дБ, необходимо обеспечить снижение усиления тракта до:
\begin{equation}
121.05 - 26.98 = 94.07\ \text{дБ}
\end{equation}

Демодулятор допускает изменение усиления в пределах от 0 до 15 дБ, а VGA (AD8338) — от 0 до 80 дБ, что позволяет гибко управлять усилением в полном необходимом диапазоне.

Таким образом, схема с двумя каскадами видеотракта обеспечивает как максимальное усиление с запасом, так и необходимую регулировку усиления в пределах всего требуемого динамического диапазона приёмника.

Управляющее напряжение $u_y$, поступающее на АРУ, формирует соответствующий режим работы активных элементов тракта по постоянному току, обеспечивая динамическую адаптацию усиления в зависимости от уровня входного сигнала.














\newpage

\section{Список используемой литературы}
\begin{enumerate}

\item 	3GPP TS 25.101 Technical Specification Group Radio Access Network
\item  Клич С.М., Кривенко А.С., Носикова Г.Н. и др., Проектирование радиоприёмных устройств: Учебное пособие для вузов / Под ред. А.П. Сиверс. –  М.: Советское радио, 1976 
\item  Логвинов В.В. Радиоприёмные устройства систем мобильной связи: Учебно-методическое пособие – М.: МТУСИ, 2016
\item  Косичкина Т.П. Курсовое проектирование радиоприёмных устройств для телерадиовещания: Учебно-методическое пособие – М.: МТУСИ, 2018
\item  Фомин Н.Н., Буга Н.Н., Головин О.В., и др., Радиоприемные устройства: Учебник для вузов / Под ред. Н.Н.Фомина. – М.: Горячая линия –Телеком, 2007.
\item  Пестряков А.В. Проектирование радиоприёмных устройств мобильной связи: Практические занятия, Москва, 2024.

\item  \href{https://en.lntwww.de/Examples_of_Communication_Systems/UMTS_Network_Architecture?utm_source=chatgpt.com}{UMTS Network Architecture}

\item  \href{https://www.alldatasheet.com/datasheet-pdf/view/1418778/INFINEON/BGA9H1MN9.html}{Datasheet for LNA - Infineon BGA9H1MN9}

\item  \href{https://www.alldatasheet.com/datasheet-pdf/view/565351/APITECH/SF1880BA02524S.html}{Datasheet for SAW filter - SF1880BA02524S}

\item  \href{https://www.alldatasheet.com/html-pdf/902731/AD/ADRF6820ACPZ-R7/70/2/ADRF6820ACPZ-R7.html}{Datasheet for I/Q Demodulator  - ADRF6820}


\item  \href{https://www.everythingrf.com/products/duplexers/anatech-electronics/690-30-am1880-1960d1004}{Datasheet for Duplexer   - AM1880-1960D1004}


\item  \href{https://www.alldatasheet.com/html-pdf/265342/LINER/LTC6603/306/2/LTC6603.html}{Datasheet for Low-Pass Ffilter with amplification - LTC6603}


\item  \href{https://www.alldatasheet.com/datasheet-pdf/view/513439/AD/AD8338.html}{Datasheet for variable gain amplifier - AD8338}

\item  \href{https://www.alldatasheet.com/datasheet-pdf/view/48122/AD/AD6600.html}{Datasheet for ADC - AD6600}



\end{enumerate}



 \newpage

\section{Приложениe}
\begin{enumerate}

\item  МШУ  \\
\href{https://www.alldatasheet.com/datasheet-pdf/view/1418778/INFINEON/BGA9H1MN9.html}{Datasheet for LNA Infineon BGA9H1MN9}
\begin{figure}[H]
    \centering
    \includegraphics[width=1\linewidth]{BGA9H1MN9.png}
    \label{fig:enter-label}
\end{figure}

\newpage

\item  SAW фильтр \\
\href{https://www.alldatasheet.com/datasheet-pdf/view/565351/APITECH/SF1880BA02524S.html}{Datasheet for SAW filter - SF1880BA02524S}
\begin{figure}[H]
    \centering
    \includegraphics[width=1\linewidth]{SF1880.png}
    \label{fig:enter-label}
\end{figure}


\newpage

\item  Демодулятор \\
\href{https://www.alldatasheet.com/html-pdf/902731/AD/ADRF6820ACPZ-R7/70/2/ADRF6820ACPZ-R7.html}{Datasheet for I/Q Demodulator  - ADRF6820}
\begin{figure}[H]
    \centering
    \includegraphics[width=1\linewidth]{ADRF6820.png}
    \label{fig:enter-label}
\end{figure}

\newpage

\item  Дуплексер \\
\href{https://www.everythingrf.com/products/duplexers/anatech-electronics/690-30-am1880-1960d1004}{Datasheet for Duplexer  filter - AM1880-1960D1004}
\begin{figure}[H]
    \centering
    \includegraphics[width=1\linewidth]{duplexer.png}
    \label{fig:enter-label}
\end{figure}

\newpage

\item  ФНЧ с усилением \\
\href{https://www.alldatasheet.com/html-pdf/265342/LINER/LTC6603/306/2/LTC6603.html}{Datasheet for Low-Pass Ffilter with amplification - 
LTC6603}
\begin{figure}[H]
    \centering
    \includegraphics[width=1\linewidth]{LTC6603.png}
    \label{fig:enter-label}
\end{figure}

\newpage

\item  Усилитель с переменным усилением \\
\href{https://www.alldatasheet.com/datasheet-pdf/view/513439/AD/AD8338.html}{Datasheet for variable gain amplifier - AD8338}
\begin{figure}[H]
    \centering
    \includegraphics[width=1\linewidth]{AD8338.png}
    \label{fig:enter-label}
\end{figure}

\newpage

\item  АЦП \\
\href{https://www.alldatasheet.com/datasheet-pdf/view/48122/AD/AD6600.html}{Datasheet for ADC - AD6600}
\begin{figure}[H]
    \centering
    \includegraphics[width=1\linewidth]{AD6600.png}
    \label{fig:enter-label}
\end{figure}

\newpage



\end{enumerate}


\end{document}


